\chapter{Consciousness}

Where is consciousness?
Some say it's in the brain.
So the entirety of the world around us, the birds, the trees, the mountains, the sky, is contained entirely within our brains?
Would our brains be capable of holding all that information and rendering it in real-time?
There exists somewhere in one's mind a model of the external world; a representation of where things are in relation to each other (\textit{i.e.} your favourite route around the park, or where you normally leave your keys) and how things interact with each other (\textit{e.g.} things fall to the ground, water is wet) strong enough that it can cause significant discontent when it is wrong (\textit{e.g.} when you can't find your keys, or when you trip over something unexpected).
But does this model exist \textit{in your brain}, or is it spread out amongst the things which it represents and explains?

Perhaps the consciousness is not in the brain, but rather is distributed in the relevant regions of space for the present concern.
The interpolation idea comes from materialism and physicalism: particles, fields, are unconscious in the equations of physics.
But then we are led to believe that consciousness is a special emergent property of life, and we are conscious beings in a purely unconscious universe.
Why should we, or life on earth in general, be so special?
In fact why stop there, maybe only \textit{you} are conscious, and everone else's brain activity is just a delicate dance of unconscious particles.
This seems a bit far-fetched, doesn't it, and also quite solipsistic. It would be muich easier for \textit{everything} to be conscious, and for consciousness to arise from the interactions between things.

So, the non-negligible consciousness of the brain arises from the vast number of connections between the neurons.
These connections selectively allow the passage of information between neurons, and when many are fed into each other to create a network, complex patterns can emerge and in some manner our conscious experience arises from this.
But where does the consciousness come from?
At some point, the network of neurons transitions from being a collection of cells to being a conscious entity.
This does not seem likely at all.
For there to be a transition from non-conscious to conscious, this implies a somewhat Boolean view of consciousness: either something is conscious, or it is not.
Are we more conscious than a monkey?
Or a dog?
Or a worm?
Or a bacterium?
Clearly at some point we may say that something is not conscious, but perhaps what we really mean is that it is of negligible consciousness.
Its behaviour is not significantly affected by its consciousness, and so for practical purposes we can treat it as non-conscious.
It then follows that consciousness really ought to be treated continuously, and this raises many questions.
Is consciousness related to intelligence?
Or knowledge?
Is learning a new skill, or meeting a new person, a consciousness-gaining experience?

It has become quite apparent to me that everything is conscious.
Or rather, consciousness exists in every interaction between things --- the more energy which is transferred in the interaction (i.e. the degree to which the two particles change state), the more powerful the consciousness is, because more information is being exchanged.
The more connections you have, the more consciousness you have.
% is the amount of consciousness also determined by the complexity of the interaction? Or the speed of information transfer between the nodes? or is consciousness just a measure of number of conections?

So, maybe learning things does affect consciousness.
It is implicitly familiar that the extent of consciousness can change in more tangible cases, e.g. when using a familiar tool.
The feeling of something 'becoming a third arm' clearly alludes to this.

So then, every interaction has consciousness.
That therefore means that every system of interacting objects has consciousness.
A rock is a system of atoms which interact.
But since the atoms are bound in tight potentials, unable to effectively exchange information with each other, the overall consciousness of the rock is very low.
Liquids can interact and so water molecules have a higher degree of information exchange than the atoms in the rock, and so we might consider water having a higher degree of consciousness than a rock.
Plant cells interact with each other and their environment much more freely, and in some cases with the plants actually driving reactions which benefit them, for example the uptake of water.
Able to adapt to their environment and tailor their behaviour to maximise survival, plants without doubt have a higher degree of consciousness than rocks or water.
And then we have the animals on top of that, whose sophisticated structures, which build themselves into stomachs and hearts and brains and nervous systems to improve the connection speed and quality, allow for even more complex interactions and information exchange, and are thus orders of magnitude more conscious than even the most intelligent plants. % is intelligence consiousness though?

What about the environments they inhabit? A forest is a complex system of interacting plants, animals, fungi, and more.
The concept of the mycelium network acting as a means of intra-ecosystem communication (transportation of nutrients) is well established, but many would fall short of decribing it as conscious.
As we accept that non-negligible consciousness arises from ordered interactions, it seems inevitable that the mycelium network is conscious, and with it the entire forest ecosystem as a whole.

What about the external envonironment?
Consider global weather systems, which exhibit complex ordered behaviour, with feedback loops, interactions between different components, and higher-order emergent phenomena from interaction between those things.
The amount of information which can be encoded in the weather system is vast, and it is not a jump to imagine storing information in it: the temperatures of the previous year, for example, are to some degree (weakly) encoded in the current state of the El Nino system.
So the weather system has a memory, and memory certainly feels like a feature of consciousness.
So if the weather system is conscious, can we make it higher or lower?
Climate change is disrupting the natural patterns of the weather, which is in effect decreasing the strength of the coupling between different parts of the system.
The chaos which is being inserted into the system is reducing the amount of ordered information.
Is climate change decreasing the consciousness of the atmosphere?

So, the hard work is done.
It does not take much to extend this line of reasoning to the solar system, the galaxy, the universe, which must all be conscious to some degree.
On the largest scale, the universe is a web of filaments, along which information can certainly be transferred.

The catch here is the speed of information transfer.
On the scale of neurons in the brain, information transfer is effectively instantaneous.
Information transfer through hormones is slower, but still fast enough to be relevant on the timescale of seconds to minutes.
Transfer of nutrients in ecosystems happens over years, and weather systems exhibit patterns on annual scales.

Galaxy filaments exhibit lengths in the range of 200 million light years---so any consciousness which arises from information transfer along these filaments (of which there are perhaps 50 million in the observable universe) is certainly very slow.
Taking the product of the number of connections, $n$, and the inverse time of information transfer per connection, $\nu = \frac{v}{d}$, we can get a feel for the degree of processing speed of a system as $C = n \cdot \nu$.
For the universe, we have $\nu =$ \qty{1e-16}{\per\second}, and thus $C =$ \qty{1e-8}{\per\second}.
For a brain, we have $\nu =$ \qty{1e6}{\per\second} (assuming neurons of \qty{1}{\milli\meter} in length) and thus $C =$ \qty{1e17}{\per\second} for the 90 billion neurons in the human brain.
It is little wonder, therefore, that the universe feels so inanimate compared to our own conscious experience.
Whether this makes the universe \textit{less} conscious, or perhaps less intelligent, or both (or perhaps they are synonymous), is unclear.
%I think it makes most logical sense for consciousness to come from number of connections, and intelligence to come from processing speed. Because then at least intelligence is derived from consciousness. but it doesn't feel fair, I think there is a difference between being quicker to think and being more conscious (i.e. being able to see patterns that others cannot, no matter how long they think about them, for example).

