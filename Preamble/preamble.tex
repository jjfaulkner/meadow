\ifsetCustomMargin
  \RequirePackage[left=37mm,right=30mm,top=35mm,bottom=30mm]{geometry}
  \setFancyHdr % To apply fancy header after geometry package is loaded
\fi

\raggedbottom

% To remove the excess top spacing for enumeration, list and description
%\usepackage{enumitem}
%\setlist[enumerate,itemize,description]{topsep=0em}

% Captions: This makes captions of figures use a boldfaced small font.
\RequirePackage[small,bf,width=0.8\textwidth]{caption}

\RequirePackage[labelsep=space,tableposition=top]{caption}
\renewcommand{\figurename}{Fig.} %to support older versions of captions.sty

\usepackage{subcaption}

\usepackage{amssymb}           % extra maths symbols
\usepackage{amsmath}           % aligning equations
\usepackage[version=4]{mhchem} % chemical equations
\usepackage[super]{nth}        % neat ordinal numbers
\usepackage{mathrsfs}          % nice swirly letters. works with \mathscr{text}
\usepackage{braket}            % for easy Dirac notation
\usepackage{amsthm}
\usepackage{nomencl}
\renewcommand\nomgroup[1]{%
  \item[\bfseries
  \ifstrequal{#1}{X}{Acronyms and Abbreviations}{%
  \ifstrequal{#1}{A}{Roman Symbols}{%
  \ifstrequal{#1}{G}{Greek Symbols}{}}}%
]}
\renewcommand{\nomname}{List of Acronyms and Symbols}
\makenomenclature

\usepackage{ifthen}
\usepackage{etoolbox}

\newtheorem{theorem}{Theorem}[section]

\usepackage{upgreek}           % upright greek letters
\usepackage[greek,english]{babel}

\usepackage{float}
\usepackage{pgfplots}
\pgfplotsset{compat=1.18}
\usepackage{tikz}
\usepackage{graphicx}
\usepackage{array} % for better options with tables (p column type)
\renewcommand{\arraystretch}{1.5} % for more space in tables

\usepackage{bm}
\usepackage{microtype}         % imporoves text justification

\usepackage{booktabs} % For professional looking tables
\usepackage{multirow}

\usepackage{rotating}

%\usepackage{multicol}
%\usepackage{longtable}
%\usepackage{tabularx}

\usepackage{siunitx}           % SI units
\sisetup{inter-unit-product=\ensuremath{{}\cdot{}}}
\sisetup{separate-uncertainty=true}
\DeclareSIUnit\angstrom{\text {Å}}
\DeclareSIUnit\rydberg{\text{Ry}}
\DeclareSIUnit\bar{bar}
\DeclareSIUnit\torr{Torr}
% ************************ Formatting / Footnote *******************************

% Don't break enumeration (etc.) across pages in an ugly manner (default 10000)
%\clubpenalty=500
%\widowpenalty=500

%\usepackage[perpage]{footmisc} %Range of footnote options

% Add `custombib' in the document class option to use this section
\ifuseCustomBib
   \RequirePackage[square, sort, numbers, authoryear]{natbib} % CustomBib

% If you would like to use biblatex for your reference management, as opposed to the default `natbibpackage` pass the option `custombib` in the document class. Comment out the previous line to make sure you don't load the natbib package. Uncomment the following lines and specify the location of references.bib file

%\RequirePackage[backend=biber, style=numeric-comp, citestyle=numeric, sorting=nty, natbib=true]{biblatex}
%\bibliography{References/references} %Location of references.bib only for biblatex

\fi

% changes the default name `Bibliography` -> `References'
\renewcommand{\bibname}{References}
\newcommand{\hb}{\noindent\rule{\textwidth}{0.5pt}}
\newcommand*{\hham}{\hat{\mathcal{H}}} % Hamiltonian operator

% ****** TOC depth and numbering depth ****

\setcounter{secnumdepth}{3}
\setcounter{tocdepth}{2}

% *********************** Configure Draft Mode **********************************

% Uncomment to disable figures in `draft'
%\setkeys{Gin}{draft=true}  % set draft to false to enable figures in `draft'

% These options are active only during the draft mode
% Default text is "Draft"
%\SetDraftText{DRAFT}

% Default Watermark location is top. Location (top/bottom)
%\SetDraftWMPosition{bottom}

% Draft Version - default is v1.0
%\SetDraftVersion{v1.1}

% Draft Text grayscale value (should be between 0-black and 1-white)
% Default value is 0.75
%\SetDraftGrayScale{0.8}

% ******************************** Todo Notes **********************************
%% Uncomment the following lines to have todonotes.

%\ifsetDraft
%	\usepackage[colorinlistoftodos]{todonotes}
%	\newcommand{\mynote}[1]{\todo[author=kks32,size=\small,inline,color=green!40]{#1}}
%\else
%	\newcommand{\mynote}[1]{}
%	\newcommand{\listoftodos}{}
%\fi

% Example todo: \mynote{Hey! I have a note}