\chapter{Balance}

% sets out the argument for peace not as passivity but as dynamic balance, and the importance of conflict in maintaining balance

Where consciousness is most concentrated, it is often considered that the most desirable state is one of peace and harmony.
Whilst this is true, the concept of what constitutes peace and harmony is often misunderstood.
It is not about universal passivity; it is not about the absence of conflict; it is not about ubiquitous agreement.
Rather it is about something far more beautiful: \textit{balance}.

Balance is a dynamic equilibrium.
It is the stability of an ecosystem in the presence of a new challenge, not without suffering, but with predators and prey in a constant game of survival.
It is the motion of the planets and of the stars, held in their orbits by the competing effects of gravity and momentum.
It is balance, not passivity, which leads to resilience, and it is in an appreciation of balance that we may find the emergence of peace.

The meadow is a place of endless battles.
Grasses and trees compete for sunlight, water and nutrients.
They starve each other in order to survive, produce flowers and seeds.
Insects feast on the plants, taking their leaves, their nectar; they grow and reproduce, providing food for the birds and the spiders and the mammals, which may also hunt seeds, fruits and each other.
Bacteria and fungi eat what is left, breaking down the dead matter to give it new life in the next generation of plants.
Every organism is in a constant struggle, and yet what emerges is a beautiful balance, in which all things have their place.
All things contribute towards peace, not through passivity, but through dynamic interaction.
And every interaction contributes to the consciousness of the meadow.